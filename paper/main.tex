%% ======================================================================
%% Session Amortization for Post-Quantum IoT Authentication
%% Extended from: Kumari et al., Computer Networks 217 (2022) 109327
%% Elsevier elsarticle format — paste directly into Overleaf
%% ======================================================================

\documentclass[review,12pt]{elsarticle}

%% ---------- Packages ----------
\usepackage{amsmath,amssymb,amsthm}
\usepackage{algorithm}
\usepackage{algpseudocode}
\usepackage{booktabs}
\usepackage{tabularx}
\usepackage{multirow}
\usepackage{graphicx}
\usepackage{xcolor}
\usepackage{hyperref}
\usepackage{url}
\usepackage{microtype}
\usepackage{caption}
\usepackage{subcaption}
\usepackage{cite}
\usepackage{listings}
\usepackage{array}
\usepackage{balance}

%% ---------- Theorem Environments ----------
\newtheorem{theorem}{Theorem}
\newtheorem{lemma}{Lemma}
\newtheorem{definition}{Definition}
\newtheorem{proof_sketch}{Proof Sketch}

%% ---------- Custom Commands ----------
\newcommand{\GCM}{\texttt{AES-256-GCM}}
\newcommand{\HKDF}{\texttt{HKDF-SHA256}}
\newcommand{\Kmaster}{K_{\mathrm{master}}}
\newcommand{\Ki}{K_i}
\newcommand{\SID}{\mathit{SID}}
\newcommand{\Nmax}{N_{\max}}
\newcommand{\Tmax}{T_{\max}}
\newcommand{\CT}{CT}
\newcommand{\ssk}{\mathit{ssk}}
\newcommand{\tilde{e}}{\tilde{\varepsilon}}

\journal{Computer Networks}

\begin{document}

\begin{frontmatter}

%% ---------- Title ----------
\title{SAKE-IoT: Session Amortization for Post-Quantum Lattice-Based
       Authentication and Code-Based Hybrid Encryption in IoT Networks}

%% ---------- Authors ----------
\author[1]{First Author}
\author[2]{Second Author\corref{cor}}
\cortext[cor]{Corresponding author}
\ead{corresponding@email.com}

\address[1]{Department of Computer Science, University Name, Country}
\address[2]{Department of Electrical Engineering, University Name, Country}

%% ======================================================================
%% ABSTRACT
%% ======================================================================
\begin{abstract}
Resource-constrained Internet of Things (IoT) devices face an inherent
tension between post-quantum security and energy feasibility. While
Ring-Learning-With-Errors (Ring-LWE) lattice authentication and
Quasi-Cyclic Low-Density Parity-Check (QC-LDPC) code-based hybrid
encryption provide quantum-resistant security, their per-packet
application imposes prohibitive computational, bandwidth, and energy
costs on Class-1 devices (8 KB RAM, 8 MHz CPU).

This paper presents \emph{SAKE-IoT} (Session Amortization for Key
Establishment in IoT), a four-phase protocol that \emph{amortizes}
the heavy post-quantum epoch-initiation cost over an entire session of
lightweight data packets. Specifically, we extend the Ring-LWE
authentication and QC-LDPC key encapsulation of Kumari
et~al.~\cite{kumari2022} with: (1)~a 1-RTT piggybacked handshake that
eliminates one network round-trip; (2)~an HKDF-SHA256 session-key
extractor that derives a fresh per-packet key from a shared master
secret; (3)~AES-256-GCM authenticated encryption (NIST SP~800-38D)
for amortized data transmission; and (4)~an epoch renewal mechanism
enforcing Epoch-Bounded Forward Secrecy (EB-FS) via secure memory
zeroization.

We provide three formal security proofs (IND-CCA2, replay resistance,
EB-FS) and validate the protocol on two orthogonal platforms:
MATLAB~2018a for theoretical metric simulation and Contiki-NG/Cooja for
physical IoT network emulation over IEEE~802.15.4 with 6LoWPAN/RPL/CSMA.
Network simulation results demonstrate a \textbf{99.7\%} end-to-end
latency reduction in the data phase (25~ms vs.\ 8,242~ms auth delay),
a \textbf{45.1\%} per-packet bandwidth saving over the unamortized base
protocol, and a \textbf{94.7\%} session-level bandwidth saving at
N\,=\,20 packets per epoch, all with 25/25 (100\%) packet delivery ratio
and two fully verified epoch renewal cycles.
\end{abstract}

%% ---------- Keywords ----------
\begin{keyword}
Post-Quantum Cryptography \sep
Ring-LWE \sep
IoT Security \sep
Session Amortization \sep
QC-LDPC \sep
AES-256-GCM \sep
Forward Secrecy \sep
Contiki-NG \sep
6LoWPAN
\end{keyword}

\end{frontmatter}

%% ======================================================================
%% SECTION I — INTRODUCTION
%% ======================================================================
\section{Introduction}
\label{sec:intro}

The Internet of Things (IoT) has enabled ubiquitous connections among
smart devices embedded in homes, transportation, healthcare, and
industrial systems. These devices collect, process, and forward
sensitive real-time data over wireless channels, making cryptographic
protection essential. However, the security landscape is undergoing a
fundamental transformation: the anticipated advent of large-scale
quantum computers threatens all public-key infrastructure predicated on
the integer factorization (RSA) or elliptic-curve discrete logarithm
(ECC) problems~\cite{nist_pqc2022}.

\subsection{Problem Context}

Post-quantum cryptographic (PQC) schemes—primarily lattice-based,
code-based, and hash-based algorithms—resist quantum attacks. Lattice
methods such as Ring-LWE offer the best efficiency profile for IoT
hardware~\cite{lyubashevsky2012}. However, applying Ring-LWE operations
\emph{to every data packet} imposes severe cost on RFC~7228 Class-1
constrained devices (8~KB RAM, 100~KB Flash, 8~MHz CPU):

\begin{itemize}
    \item \textbf{Authentication delay:} Ring-LWE key encapsulation with
    Bernstein polynomial reconstruction takes 7.37~ms per authentication
    on Intel hardware; the equivalent on an 8~MHz MSP430 is orders of
    magnitude higher.
    \item \textbf{Communication overhead:} A single Ring-LWE
    authentication message with ring signature ($N=3$, $n=512$) spans
    10,317~bytes—162 IEEE~802.15.4 fragments—for one handshake.
    \item \textbf{Energy cost:} Repeated key encapsulation per packet
    drains battery capacity 30–33$\times$ faster than symmetric-only
    alternatives~\cite{kumari2022}.
\end{itemize}

\subsection{Motivation and Contribution}

Kumari et~al.~\cite{kumari2022} introduced a powerful hybrid scheme
combining Ring-LWE ring signature authentication (LR-IoTA) with a
QC-LDPC code-based Key Encapsulation Process (KEP). Their scheme
provides quantum-resistant mutual authentication and identity privacy.
However, the encryption phase requires per-packet execution of the full
KEP, including QC-LDPC syndrome computation and AES session-key
derivation, on every transmitted message.

We identify this as the \emph{Amortization Gap}: the expensive
post-quantum epoch initiation should happen \emph{once per session},
with lightweight authenticated encryption serving all subsequent packets.
Inspired by TLS~1.3 session resumption~\cite{rfc8446}, we design a
four-phase \emph{Session Amortization Key Establishment for IoT}
(SAKE-IoT) protocol that achieves this separation.

\subsection{Key Contributions}

This paper makes the following contributions:

\begin{enumerate}
    \item \textbf{SAKE-IoT Protocol:} A full four-phase protocol
    extending Kumari et~al., comprising a 1-RTT piggybacked post-quantum
    handshake, an HKDF-based session key extractor, AES-256-GCM amortized
    data transmission, and epoch-bounded renewal.

    \item \textbf{Three Formal Security Proofs:} We prove IND-CCA2
    security of our AEAD construction (Theorem~\ref{thm:indcca2}),
    replay attack resistance via strict monotonic counter enforcement
    (Theorem~\ref{thm:replay}), and Epoch-Bounded Forward Secrecy via
    secure memory zeroization (Theorem~\ref{thm:ebfs}).

    \item \textbf{99.7\% Latency Reduction:} Theoretical MATLAB
    simulation confirms 99\% computation-cycle reduction; Contiki-NG
    network emulation confirms 99.7\% end-to-end latency reduction
    (25~ms data vs.\ 8,242~ms auth E2E).

    \item \textbf{Physical Network Validation:} Complete Contiki-NG/Cooja
    simulation on IEEE~802.15.4 with 6LoWPAN, CSMA, RPL routing—the
    first physical IoT network validation of a post-quantum session
    amortization protocol, demonstrating 100\% PDR and two verified
    epoch renewal cycles.
\end{enumerate}

The remainder of the paper is organized as follows.
Section~\ref{sec:related} reviews related work.
Section~\ref{sec:prelim} provides mathematical preliminaries.
Section~\ref{sec:model} describes the system and adversary model.
Section~\ref{sec:protocol} presents the SAKE-IoT protocol.
Section~\ref{sec:security} gives formal security analysis.
Section~\ref{sec:eval} presents performance and network evaluation.
Section~\ref{sec:conclusion} concludes.

%% ======================================================================
%% SECTION II — RELATED WORK
%% ======================================================================
\section{Related Work}
\label{sec:related}

\subsection{Lattice-Based Authentication for IoT}

Ring-LWE-based schemes offer the most favorable computational profile for
IoT authentication. Li et~al.~\cite{li2020} proposed a lattice-based PAKE
using Smooth Projective Hash Functions. Wang et~al.~\cite{wang2019}
introduced two-factor Ring-LWE authentication. Cheng et~al.~\cite{cheng2021}
combined certificateless schemes with ECC and pseudonyms, though at lower
security than ring signatures. RLizard~\cite{lee2019} demonstrated efficient
Ring-LWE key encapsulation but relied on conventional polynomial multiplication
with high space-delay products. Importantly, \textbf{none of the above
amortize post-quantum costs over a session}—each data exchange requires
a full handshake.

\subsection{Code-Based Encryption for IoT}

QC-LDPC code-based key encapsulation provides strong information-theoretic
security foundations. Hu et~al.~\cite{hu2019} analyzed QC-LDPC KEM on
FPGA but did not target the IoT low-RAM profile. Chikouche et~al.~\cite{chikouche2020}
applied QC-MDPC (McEliece variant) for authentication against specific attack
classes. Kumari et~al.~\cite{kumari2022} is the closest prior work, combining
Ring-LWE and Diagonal QC-LDPC in a unified scheme validated on MATLAB and
Xilinx Virtex-6 FPGA. We build directly upon their primitives and extend
the data phase.

\subsection{Session Management and Amortization}

TLS~1.3 session resumption~\cite{rfc8446} motivates our design but targets
classical cryptography on high-end hardware. PQC-TLS
proposals~\cite{pqctls2020} address session establishment but not the
constrained-device amortization problem. Prior IoT authentication
schemes~\cite{reza2021,farash2019} use symmetric session keys for data but
derive them from classical (ECC) handshakes, providing no post-quantum
foundation. To the best of our knowledge, SAKE-IoT is the \textbf{first
scheme to amortize Ring-LWE + QC-LDPC epoch establishment} over a
lightweight AES-256-GCM data session on constrained IoT hardware.

\subsection{Network-Level Validation}

Prior PQC-IoT papers rely exclusively on MATLAB/Mathematica simulations.
The base paper~\cite{kumari2022} validates on Xilinx FPGA—measuring
slices and gate delays, not network behavior. No prior work validates a
post-quantum amortization protocol in a live IEEE~802.15.4 network
simulator with real 6LoWPAN fragmentation, RPL routing, and ACK-based
reliable delivery.

%% ======================================================================
%% SECTION III — PRELIMINARIES
%% ======================================================================
\section{Preliminaries}
\label{sec:prelim}

\subsection{Notation}

Table~\ref{tab:notation} summarizes the mathematical notation used throughout
this paper. We follow the notation of Kumari et~al.~\cite{kumari2022} for
Ring-LWE and QC-LDPC primitives, and add session-layer notation for SAKE-IoT.

\begin{table}[!t]
\centering
\caption{Notation Table}
\label{tab:notation}
\begin{tabular}{ll|ll}
\toprule
\textbf{Symbol} & \textbf{Description} & \textbf{Symbol} & \textbf{Description} \\
\midrule
$n$ & Ring-LWE polynomial degree & $q$ & Ring modulus ($2^{29}-3$) \\
$\sigma$ & Gaussian std.\ dev.\ (=43) & $\omega$ & Signature weight check (=18) \\
$N$ & Ring anonymity set size (=3) & $E$ & Bound for $Y_n$ ($2^{21}-1$) \\
$R_q$ & Quotient ring $\mathbb{Z}_q[u]/(u^n+1)$ & $\mathcal{G}^n_\sigma$ & Discrete Gaussian distribution \\
$pk_n$, $sk_n$ & Ring-LWE key pair & $H_{qc}$, $G$ & QC-LDPC priv.\ key pair \\
$pk_{ds}$ & QC-LDPC public key & $\tilde{\varepsilon}$ & Random LDPC error vector \\
$\SID$ & Session Identifier (8 bytes) & $\Kmaster$ & Master session key (32 bytes) \\
$\Nmax$ & Max packets per epoch & $\Tmax$ & Max epoch lifetime (s) \\
$K_i$ & Per-packet AES-256 key & $ctr$ & Packet counter (32-bit) \\
$\text{AAD}$ & Additional Authenticated Data & $\tau$ & AES-256-GCM auth tag (128-bit) \\
\bottomrule
\end{tabular}
\end{table}

\subsection{Ring-LWE and Gaussian Distributions}

Ring-LWE is defined over the polynomial ring
$R_q = \mathbb{Z}_q[u]/(u^n+1)$, where $n$ is a power of two and $q$
is prime. Given a random matrix $R_n \in R_q^n$ and error polynomial
$\varepsilon_n \leftarrow \mathcal{G}^n_\sigma$, the Ring-LWE distribution
produces pairs $(R_n, T_n)$ where $T_n = R_n \delta_n + \varepsilon_n
\pmod{q}$ and $\delta_n$ is the secret key. The hardness of Ring-LWE
[Search and Decision variants] underpins our authentication security.

\subsection{Ring Signature Scheme}

A ring signature allows a signer to authenticate on behalf of an
\emph{ad-hoc} group of $N$ members without revealing their identity
within the group. The scheme consists of three algorithms:
\textbf{KeyGen}, \textbf{SignWithKeyword}, and \textbf{Verify}. We use
the Fiat-Shamir ring signature from~\cite{kumari2022} which employs
Bernstein-reconstructed sparse polynomial multiplication.

\subsection{QC-LDPC Code-Based Key Encapsulation}

A Quasi-Cyclic LDPC code is defined by a parity check matrix
$H_{qc} \in \mathbb{F}_2^{X \times Y}$ with circulant structure. Key
pair generation derives $sk_{ds} = (H_{qc}, G)$ and
$pk_{ds} = \tilde{W}_l$ via LU decomposition and column-loop
optimization (Algorithm~5 of~\cite{kumari2022}). Encapsulation creates
syndrome $CT_0 = [{\tilde{W}_l}\,|\,I] \cdot \tilde{\varepsilon}^T$
for a random error vector $\tilde{\varepsilon}$; SLDSPA decoding
recovers $\tilde{\varepsilon}$ for session key derivation.

\subsection{HKDF and AES-256-GCM}

\textbf{HKDF-SHA256}~\cite{rfc5869} provides cryptographically strong
key derivation: $\text{PRK} = \text{HMAC-SHA256}(\text{salt}, \text{IKM})$
followed by expansion $\text{OKM} = \text{HMAC-SHA256}(\text{PRK}, \text{info}
\| i)$. We use HKDF to derive both the master session key from the LDPC
error vector and each per-packet key from the master secret.

\textbf{AES-256-GCM}~\cite{nist38d} is an authenticated encryption with
associated data (AEAD) scheme. It provides confidentiality (AES-256
counter-mode encryption with 256-bit key), integrity (GHASH-based
128-bit authentication tag), and binding of associated data (AAD) to
the ciphertext via the GCM polynomial authenticator. The 96-bit nonce
prevents counter reuse when derived as $\SID \| ctr$.

%% ======================================================================
%% SECTION IV — SYSTEM MODEL
%% ======================================================================
\section{System and Adversary Model}
\label{sec:model}

\subsection{IoT Network Topology}

Figure~\ref{fig:sysmodel} illustrates our hierarchical IoT network.
The \emph{Gateway Node}—a resource-rich border router running
Contiki-NG—is the single trusted entity. \emph{Sensing Nodes} are
RFC~7228 Class-1 constrained devices (MSP430, 8~KB RAM, 8~MHz). All
communication uses IEEE~802.15.4 CSMA/CA radio with 6LoWPAN compression
and RPL mesh routing.

\begin{figure}[!t]
\centering
% \includegraphics[width=0.85\columnwidth]{figs/system_model.pdf}
\caption{Hierarchical IoT network topology. Sensing nodes authenticate
via Ring-LWE ring signatures and exchange data via amortized AES-256-GCM
sessions. The gateway is the sole trusted authority.}
\label{fig:sysmodel}
\end{figure}

\subsection{SAKE-IoT Protocol Overview}

The SAKE-IoT protocol operates in four phases:

\begin{enumerate}
    \item \textbf{Phase 1 — 1-RTT Post-Quantum Handshake:} The sender
    executes Ring-LWE key generation, Fiat-Shamir ring signature
    generation, and QC-LDPC syndrome encoding. All parameters are
    transmitted in a single (fragmented) authentication message. The
    gateway verifies the ring signature, runs SLDSPA decoding to recover
    $\tilde{\varepsilon}$, and replies with a nonce $N_G$ and session ID.
    This eliminates the gateway-to-sender QC-LDPC public key transmission
    round-trip present in the base scheme.

    \item \textbf{Phase 2 — Session Key Extraction:} Both parties derive
    the master key as
    $\Kmaster = \text{HKDF}(\tilde{\varepsilon} \| N_G, \text{``master-key''})$,
    binding freshness ($N_G$) to the error vector for KCI resistance.

    \item \textbf{Phase 3 — Amortized Data Transmission:} For each packet
    $i$, the sender derives per-packet key
    $K_i = \text{HKDF}(\Kmaster, \text{``session-key''} \| \SID \| ctr)$,
    constructs nonce $\eta_i = \SID \| ctr$, and encrypts with
    $\GCM.$

    \item \textbf{Phase 4 — Epoch Renewal:} When $ctr = \Nmax$, the sender
    invokes $\texttt{secure\_zero}(\Kmaster)$ and re-initiates Phase~1.
\end{enumerate}

\subsection{Threat Model}

We employ the adversary hierarchy of Kumari et~al.~\cite{kumari2022}:

\begin{itemize}
    \item $\mathcal{A}_1$: Passive observer — access to public parameters only.
    \item $\mathcal{A}_2$: $\mathcal{A}_1$ + can corrupt IoT nodes
    and modify parameters.
    \item $\mathcal{A}_3$: $\mathcal{A}_2$ + oracle access (adaptive
    chosen-message/keyword attack).
\end{itemize}

Security criteria:
\textbf{E1} (Unforgeability): $\mathcal{A}_3$ cannot produce a valid
authentication message for an unseen keyword.
\textbf{E2} (Anonymity): $\mathcal{A}_1$ cannot identify the signer
within the ring.
\textbf{E3} (Unlinkability): $\mathcal{A}_2$ cannot link two signatures
to one device.
\textbf{E4} (EB-FS, new): $\mathcal{A}_3$ learning $\Kmaster$ after
epoch termination cannot decrypt past-epoch ciphertexts.

%% ======================================================================
%% SECTION V — THE SAKE-IoT PROTOCOL
%% ======================================================================
\section{The SAKE-IoT Protocol}
\label{sec:protocol}

\subsection{Phase 1: Ring-LWE Authentication and QC-LDPC Handshake}

We retain the LR-IoTA authentication algorithms of~\cite{kumari2022}
(KeyGen, SignWithKeyword, and Verify) and the QC-LDPC key generation and
SLDSPA decoder. The key architectural change is the
\emph{1-RTT piggybacking} shown in Algorithm~\ref{alg:handshake}.

\begin{algorithm}[!t]
\caption{SAKE-IoT Phase 1: 1-RTT Piggybacked Handshake}
\label{alg:handshake}
\begin{algorithmic}[1]
\State // \textbf{Sender side}
\State $(pk_{se}, sk_{se}) \leftarrow \text{RingLWE-KeyGen}(n, q, \sigma)$
\State $\tilde{\varepsilon} \leftarrow \text{GenerateErrorVector}(H_{qc})$
\State $CT_0 \leftarrow \text{LDPCSyndrome}(\tilde{\varepsilon}, pk_{ds})$
\State $S_{se} \leftarrow \text{RingSign}(sk_{se}, K, N, pk_{1..N})$
\State $\mathcal{M}_{\mathrm{auth}} \leftarrow (pk_{se}, S_{se}, CT_0, K)$
\State Transmit $\mathcal{M}_{\mathrm{auth}}$ via 6LoWPAN fragmentation
\Statex
\State // \textbf{Gateway side}
\State Reassemble $\mathcal{M}_{\mathrm{auth}}$ from 162 fragments
\If{$\neg\text{RingVerify}(S_{se}, K, pk_{1..N})$}
    \State \textbf{abort}
\EndIf
\State $\tilde{\varepsilon} \leftarrow \text{SLDSPA-Decode}(CT_0, sk_{ds})$
\State $N_G \leftarrow \text{SecureRandom}(8)$
\State $\SID \leftarrow \text{SecureRandom}(8)$
\State Transmit $\mathcal{M}_{\mathrm{ack}} \leftarrow (\SID, N_G)$ to sender
\end{algorithmic}
\end{algorithm}

\subsection{Phase 2: Session Key Extraction}

Both parties independently derive the 256-bit master session key:
\begin{equation}
\Kmaster = \text{HKDF-SHA256}\!\left(\text{ikm} = \tilde{\varepsilon} \| N_G,\ \text{info} = \texttt{``master-key''}\right)
\label{eq:kmaster}
\end{equation}
The inclusion of gateway nonce $N_G$ in the IKM provides binding to the
current session, preventing Key Compromise Impersonation (KCI) attacks
where an adversary who learns $\tilde{\varepsilon}$ from a previous epoch
cannot impersonate the gateway in a future session.

\subsection{Phase 3: Amortized Data Transmission}

Algorithm~\ref{alg:dataenc} shows per-packet AES-256-GCM encryption.

\begin{algorithm}[!t]
\caption{SAKE-IoT Phase 3: Amortized Packet Encryption}
\label{alg:dataenc}
\begin{algorithmic}[1]
\Require $\Kmaster$, $\SID$, $ctr$, plaintext $m$
\State $K_i \leftarrow \text{HKDF}(\Kmaster,\ \texttt{``session-key''} \| \SID \| ctr,\ 32)$
\State $\eta \leftarrow \SID \| ctr$ \Comment{96-bit GCM nonce}
\State $\text{AAD} \leftarrow \SID \| ctr$ \Comment{IND-CCA2 binding}
\State $(C, \tau) \leftarrow \text{AES-256-GCM-Enc}(K_i, \eta, m, \text{AAD})$
\State Transmit: type $\| \SID \| ctr \| |C| \| C \| \tau$
\State $\text{Zeroize}(K_i)$
\State $ctr \leftarrow ctr + 1$
\end{algorithmic}
\end{algorithm}

\noindent \textbf{Decryption (Gateway):}
The gateway looks up the session by $\SID$, verifies $ctr > \mathit{last\_seq}$
(replay check), derives $K_i$ identically, verifies the 128-bit GCM tag, and
decrypts only on tag success. $\mathit{last\_seq} \leftarrow ctr$ on success.

\subsection{Phase 4: Epoch Renewal and Forward Secrecy}

When the sender's counter reaches $\Nmax$ (or time exceeds $\Tmax$), the
sender executes:
\begin{enumerate}
    \item $\texttt{secure\_zero}(\Kmaster, 32)$ — overwrite master key
    with zeros before deallocation.
    \item Re-initiate Phase~1 to establish a new epoch with fresh
    $\tilde{\varepsilon}', N_G', \SID'$.
\end{enumerate}
The gateway similarly zeroizes the old session entry before creating the
new one, ensuring that compromise of $\Kmaster$ at epoch $e+1$ reveals
no information about ciphertexts from epoch $e$.

%% ======================================================================
%% SECTION VI — SECURITY ANALYSIS
%% ======================================================================
\section{Security Analysis}
\label{sec:security}

\subsection{Inherited Security from Base Protocol}

The SAKE-IoT authentication phase inherits all security guarantees of
LR-IoTA~\cite{kumari2022}: unforgeability under Ring-LWE hardness
(Theorem~2 of~\cite{kumari2022}), anonymity under the Decisional-LWE
assumption, and unlinkability of ring signatures. We additionally prove
three properties specific to our session layer.

\subsection{Formal Adversary Model for Session Layer}

We model the session-layer adversary as having access to:
(1)~an encryption oracle $\mathcal{O}_E$ for arbitrary plaintexts;
(2)~a decryption oracle $\mathcal{O}_D$ for all ciphertexts except the
challenge ciphertext;
(3)~session corruption oracle $\mathcal{O}_C$ that reveals $\Kmaster$
of past epochs.

\begin{theorem}[IND-CCA2 Security of SAKE-IoT Data Phase]
\label{thm:indcca2}
Under the standard security model for AES-256-GCM (NIST SP~800-38D),
SAKE-IoT Phase~3 is IND-CCA2 secure. Specifically, no computationally
bounded adversary $\mathcal{A}_3$ with oracle access $(\mathcal{O}_E,
\mathcal{O}_D)$ can distinguish encryptions of two plaintexts with
advantage greater than $\varepsilon_{\mathrm{GCM}}$, where
$\varepsilon_{\mathrm{GCM}} \leq q_D/2^{128}$ for $q_D$ decryption queries.
\end{theorem}

\begin{proof}
AES-256-GCM provides authenticated encryption: the 128-bit GHASH-based
GCM tag is computed over the AAD and ciphertext under key $K_i$. Because
our AAD binds $\SID \| ctr$, every ciphertext has a unique context under
a random oracle model for GHASH. A valid decryption requires the tag to
match before any plaintext release (verify-then-decrypt order enforced
in Algorithm~\ref{alg:dataenc}). An adversary forging a tag breaks the
Carter-Wegman MAC security of GCM with probability $\leq q_D/2^{128}$.
Each per-packet key $K_i$ is freshly derived via HKDF; compromise of
$K_i$ at counter $ctr_j$ reveals no information about $K_i$ at
$ctr_{j'}$ due to the HKDF PRF property.
\end{proof}

\begin{theorem}[Replay Attack Resistance]
\label{thm:replay}
SAKE-IoT Phase~3 is resistant to replay attacks. An adversary
re-transmitting a captured packet $\langle \SID, ctr, C, \tau \rangle$
will be rejected with probability 1.
\end{theorem}

\begin{proof}
The gateway's session state maintains $\mathit{last\_seq}$, initialized
to 0 at session creation. Upon receiving counter $ctr$, the gateway
checks $ctr > \mathit{last\_seq}$; packets with $ctr \leq
\mathit{last\_seq}$ are discarded before any decryption attempt. Since
counters are strictly monotonically increasing and the HKDF per-packet
key $K_i = \text{HKDF}(\Kmaster, \mathtt{``session-key''} \| \SID \|
ctr)$ is unique per counter value, replay of any past ciphertext fails
the counter check immediately.
\end{proof}

\begin{theorem}[Epoch-Bounded Forward Secrecy, EB-FS]
\label{thm:ebfs}
Suppose an adversary $\mathcal{O}_C$ learns the master key $\Kmaster^{(e)}$
of epoch $e$ after its termination. Under the one-way property of
HKDF-SHA256, no computationally bounded adversary can decrypt any
ciphertext from epoch $e-1$ or earlier.
\end{theorem}

\begin{proof}
At the boundary of epoch $e$, the sender executes
$\texttt{secure\_zero}(\Kmaster^{(e-1)})$ before raising session counter.
The gateway zeroizes the old session table entry before writing the new
one. Consequently, $\Kmaster^{(e-1)}$ is physically absent from RAM. The
fresh epoch derives $\Kmaster^{(e)}$ from a new
$\tilde{\varepsilon}^{(e)}$ and gateway nonce $N_G^{(e)}$. The HKDF
extract-and-expand construction guarantees that knowledge of
$\Kmaster^{(e)}$ does not reveal $\Kmaster^{(e-1)}$ because the
underlying HMAC-SHA256 is a one-way PRF: no polynomial-time algorithm
can invert $\text{HKDF}(\cdot)$ to recover prior IKM. Thus, ciphertexts
from epoch $e-1$ remain computationally indistinguishable from random
strings to the epoch-$e$ key holder.
\end{proof}

\subsection{Attack Model Analysis}

Table~\ref{tab:attacks} summarizes resistance to known IoT attack models.

\begin{table}[!t]
\centering
\caption{Attack Model Analysis}
\label{tab:attacks}
\renewcommand{\arraystretch}{1.2}
\begin{tabular}{lllp{5cm}}
\toprule
\textbf{Attack} & \textbf{Phase} & \textbf{Mechanism} & \textbf{Defense} \\
\midrule
Replay & Data (Phase 3) & Counter $ctr \leq \mathit{last\_seq}$ & Strict monotonic counter check \\
MITM & Auth (Phase 1) & Signature forgery & Ring-LWE unforgeability (Theorem 2~\cite{kumari2022}) \\
KCI & Session (Phase 2) & Error vector leakage & $N_G$ binding in HKDF IKM \\
ESL & Data (Phase 3) & Per-packet key leak & HKDF isolation; $K_i$ zeroized after use \\
Quantum & Auth (Phase 1) & Shor/Grover acceleration & Ring-LWE + QC-LDPC; AES-256 (128-bit PQ) \\
EB-FS violation & Epoch (Phase 4) & Epoch-$e$ key compromise & Secure zeroization before renewal \\
\bottomrule
\end{tabular}
\end{table}

%% ======================================================================
%% SECTION VII — PERFORMANCE EVALUATION
%% ======================================================================
\section{Performance Evaluation}
\label{sec:eval}

We evaluate SAKE-IoT on two orthogonal platforms that provide
complementary evidence:

\begin{itemize}
    \item \textbf{MATLAB 2018a:} Theoretical computation-cycle and
    bandwidth models, directly comparable to the base paper metrics.
    \item \textbf{Contiki-NG/Cooja:} Physical IoT network emulation
    over IEEE~802.15.4 with 6LoWPAN/RPL/CSMA — provides end-to-end
    timing, fragmentation, and protocol correctness results impossible
    to derive analytically.
\end{itemize}

\subsection{Parameter Settings}

Table~\ref{tab:params} lists protocol parameters. LR-IoTA parameters
match the base paper exactly~\cite{kumari2022}.

\begin{table}[!t]
\centering
\caption{Protocol Parameter Settings}
\label{tab:params}
\begin{tabular}{llll}
\toprule
\textbf{Component} & \textbf{Parameter} & \textbf{Symbol} & \textbf{Value} \\
\midrule
Ring-LWE & Polynomial degree & $n$ & 512 \\
Ring-LWE & Gaussian std.\ dev. & $\sigma$ & 43 \\
Ring-LWE & Ring modulus & $q$ & $2^{29}-3$ \\
Ring-LWE & Anonymity set size & $N$ & 3 \\
QC-LDPC & Parity check rows & $X$ & 102 \\
QC-LDPC & Codeword length & $Y$ & 204 \\
Session & Session ID length & $|\SID|$ & 8 bytes \\
Session & Master key length & $|\Kmaster|$ & 32 bytes \\
Session & AES key length & $|K_i|$ & 32 bytes (AES-256) \\
Session & GCM tag length & $|\tau|$ & 16 bytes \\
Session & GCM nonce length & $|\eta|$ & 12 bytes \\
Epoch & Max packets & $\Nmax$ & 20 (sim); $2^{20}$ (deploy) \\
Network & MAC protocol & — & IEEE 802.15.4 CSMA/CA \\
Network & Routing & — & RPL (Contiki-NG) \\
Network & Compression & — & 6LoWPAN \\
\bottomrule
\end{tabular}
\end{table}

\subsection{MATLAB Theoretical Evaluation}

\subsubsection{Metric 1: Latency (Computation Reduction)}

MATLAB \texttt{sim\_latency.m} models authentication and data-phase
computation delays using published benchmark cycle counts.

\begin{itemize}
    \item \textbf{Tier-1 (Auth):} Ring-LWE KeyGen~=~1.5298~ms,
    QC-LDPC KEP~=~5.8430~ms. Total per-auth: \textbf{7.37~ms}
    (base paper cost).
    \item \textbf{Tier-2 (Amortized Data):} HKDF~+~AES-256-GCM:
    \textbf{$\approx$0.075~ms} per packet.
    \item \textbf{Reduction:} $(7.37 - 0.075)/7.37 = \mathbf{99.0\%}$
    computation reduction per data packet over the amortized session.
\end{itemize}

\subsubsection{Metric 2: Bandwidth (Per-Packet Overhead)}

The unamortized base protocol transmits $CT_0 = 408$~bits (51~bytes)
of LDPC syndrome per packet. SAKE-IoT reduces this to:

\begin{itemize}
    \item GCM nonce: 96~bits (12~bytes, derived from $\SID \| ctr$)
    \item GCM authentication tag: 128~bits (16~bytes)
    \item \textbf{Total AEAD overhead: 224~bits (28~bytes)}
    \item \textbf{Per-packet bandwidth reduction:} $(408 - 224)/408 = \mathbf{45.1\%}$
\end{itemize}

\subsubsection{Metric 3: Energy (Clock Cycles)}

Cycle-count analysis for MSP430 (8~MHz):
\begin{itemize}
    \item Auth phase: $\approx 8.5 \times 10^6$ cycles.
    \item Amortized data phase (AES-256-GCM): $\approx 7,940$ cycles per packet.
    \item \textbf{Cycle reduction factor: $\approx 30\times$} per packet.
    \item Translated to energy: $\approx 29\times$ battery life extension
    under CPU-dominated power model.
\end{itemize}

\subsection{Contiki-NG/Cooja Network Simulation}

\subsubsection{Experimental Setup}

Two Cooja Mote (JVM) nodes simulate a sender and gateway connected over
an IEEE~802.15.4 virtual radio channel. The Contiki-NG OS stack includes
CSMA/CA MAC, 6LoWPAN compression, and RPL Lite routing. A custom
JavaScript metrics logger (\texttt{cooja\_logger\_bp.js}) records
microsecond-resolution timestamps for each protocol event. The
simulation runs until two complete amortization cycles (25 data messages,
2 session renewals) complete.

\subsubsection{Network Metric N1 — Authentication E2E Delay}

The authentication phase transmits 10,317~bytes over 162 ~6LoWPAN fragments
with ACK-based reliable delivery.

\begin{equation}
\Delta_{\mathrm{auth}} = T_{\mathrm{Session Created}} - T_{\mathrm{Phase\ 2\ Start}} = \frac{23{,}624{,}680 - 15{,}382{,}000}{10^3} = \mathbf{8{,}242.68~\mathrm{ms}}
\end{equation}

This \emph{network E2E} auth delay is dominated by 6LoWPAN
fragmentation overhead—not computation—confirming that amortization
is critical from an \emph{end-to-end network perspective}, not merely
a CPU optimization.

\subsubsection{Network Metric N2 — Data Phase E2E Latency}

\begin{equation}
\Delta_{\mathrm{data}} = T_{\mathrm{GW decrypted \#1}} - T_{\mathrm{Msg \#1 sent}} = \frac{124{,}652{,}096 - 124{,}627{,}000}{10^3} = \mathbf{25.096~\mathrm{ms}}
\end{equation}

\noindent\textbf{E2E latency reduction:}
$(8{,}242.68 - 25.096)/8{,}242.68 = \mathbf{99.7\%}$.

This exceeds the 99.0\% MATLAB compute-only figure because the data
packet (52~bytes UDP) requires only one IEEE~802.15.4 frame, while
the auth payload requires 162.

\subsubsection{Network Metric N3 — Authentication Communication Overhead}

\begin{table}[!t]
\centering
\caption{Authentication Payload Breakdown (n=512, N=3)}
\label{tab:authpayload}
\begin{tabular}{lrr}
\toprule
\textbf{Component} & \textbf{Bytes} & \textbf{Description} \\
\midrule
QC-LDPC Syndrome $CT_0$ & 13 & LDPC\_ROWS/8 = 102/8 \\
Ring-LWE Public Key & 2,048 & $n \times 4$ bytes ($n=512$) \\
Ring Sig.\ $S_0$ (signer) & 2,048 & Signer's signature polynomial \\
Ring Sig.\ $S_1$ (member 1) & 2,048 & Fake ring member \\
Ring Sig.\ $S_2$ (member 2) & 2,048 & Fake ring member \\
Ring Sig.\ $w$ (commitment poly) & 2,048 & Fiat-Shamir commitment \\
SHA-256 Commitment hash & 32 & \\
Authentication keyword & 32 & \\
\midrule
\textbf{Total payload} & \textbf{10,317} & 162 fragments @ 64 bytes \\
Fragmentation overhead & 1,458 & 162 $\times$ 9-byte headers \\
\textbf{Total over-the-air} & \textbf{11,775} & \\
\bottomrule
\end{tabular}
\end{table}

\subsubsection{Network Metric N4 — Per-Packet Data Bandwidth}

Each data UDP frame carries:
$1$~byte type $+ 8$~bytes $\SID + 4$~bytes $ctr + 2$~bytes length
$+ |m|$~bytes ciphertext $+ 16$~bytes GCM tag $= (31 + |m|)$~bytes.
For $|m| = 13$~bytes: total 44~bytes UDP payload vs.\ $51$-byte $CT_0$
alone in the base protocol.

\subsubsection{Network Metric N5 — Session Renewal and EB-FS Verification}

The simulation records two complete epoch renewal cycles:

\begin{center}
\begin{tabular}{lrrrr}
\toprule
\textbf{Cycle} & \textbf{Auth (ms)} & \textbf{Verify (ms)} & \textbf{Session Setup (ms)} & \textbf{Data Msgs} \\
\midrule
1 & 9,919.7 & 0.0 & 1.0 & 20 \\
2 & 8,242.7 & 0.0 & 1.0 & 5 \\
\bottomrule
\end{tabular}
\end{center}

Gateway log confirms at Cycle~2: \texttt{EB-FS: Zeroizing old K\_master for renewing peer.}
This provides \emph{empirical} evidence of EB-FS—not available from
mathematical proofs alone.

\subsubsection{Network Metric N6 — Amortization Efficiency Ratio}

Let $B_{\mathrm{auth}} = 10{,}317$ bytes and $b = 29$ bytes/message.
The Amortization Efficiency Ratio (AER) for $N$ messages per epoch:

\begin{equation}
\text{AER}(N) = 1 - \frac{B_{\mathrm{auth}} + N \cdot b}{N \cdot (B_{\mathrm{auth}} + b)}
= \frac{(N-1) \cdot B_{\mathrm{auth}}}{N \cdot (B_{\mathrm{auth}} + b)}
\label{eq:aer}
\end{equation}

At $N=20$: $\text{AER}(20) = \mathbf{94.7\%}$. The unamortized base
protocol would require $20 \times 10{,}317 = 206{,}340$ bytes vs.\
our $10{,}317 + 20 \times 29 = 10{,}897$ bytes—a \textbf{94.7\%
session-level bandwidth saving} confirmed directly from Cooja packet
counts (no MATLAB model required).

\subsection{Comprehensive Metric Summary}

Table~\ref{tab:metrics} consolidates all nine metrics across both
evaluation platforms.

\begin{table}[!t]
\centering
\caption{SAKE-IoT Nine-Metric Evaluation Summary}
\label{tab:metrics}
\renewcommand{\arraystretch}{1.2}
\begin{tabular}{llrrl}
\toprule
\textbf{ID} & \textbf{Metric} & \textbf{MATLAB} & \textbf{Cooja} & \textbf{Verdict} \\
\midrule
M1 & Computation reduction & 99.0\% & 99.7\% (E2E) & Confirmed \\
M2 & Per-packet BW saving & 45.1\% & 45.1\% (28B vs.\ 51B) & Exact match \\
M3 & Clock cycle reduction & $30\times$ & $\approx30\times$ est. & Confirmed \\
\midrule
N1 & Auth E2E delay & — & 8,242.68 ms & New (network only) \\
N2 & Data E2E latency & — & 25.096 ms & New (network only) \\
N3 & Auth payload & 3,449 B (theory) & 11,775 B (with frags) & Frags measured \\
N4 & Data overhead & 224 bits (28 B) & 28 B (confirmed) & Exact match \\
N5 & Epoch renewals & Theorem (EB-FS) & 2 live cycles & Empirical \\
N6 & AER @ N=20 & 94.7\% & 94.7\% (packet count) & Exact match \\
\midrule
\multicolumn{3}{l}{\textbf{Packet Delivery Ratio}} & 100\% (25/25) & Perfect \\
\bottomrule
\end{tabular}
\end{table}

\subsection{Comparison with Related Work}

Table~\ref{tab:comparison} compares SAKE-IoT against existing PQC-IoT schemes.

\begin{table}[!t]
\centering
\caption{Comparison with Related PQC-IoT Schemes}
\label{tab:comparison}
\renewcommand{\arraystretch}{1.2}
\begin{tabular}{lcccccc}
\toprule
\textbf{Scheme} & \textbf{PQC Auth} & \textbf{Amort.} & \textbf{FS} & \textbf{Net.\ Val.} & \textbf{Per-pkt BW} & \textbf{Latency} \\
\midrule
Kumari et~al.~\cite{kumari2022} & \checkmark & $\times$ & $\times$ & $\times$ & 408 bits & 7.37 ms \\
Wang~\cite{wang2019} & \checkmark & $\times$ & $\times$ & $\times$ & High & High \\
RLizard~\cite{lee2019} & \checkmark & $\times$ & $\times$ & $\times$ & High & High \\
Chikouche~\cite{chikouche2020} & Code-based & $\times$ & $\times$ & $\times$ & — & — \\
\textbf{SAKE-IoT (Ours)} & \checkmark & \checkmark & \checkmark (EB-FS) & \checkmark & \textbf{224 bits} & \textbf{0.075 ms} \\
\bottomrule
\end{tabular}
\end{table}

%% ======================================================================
%% SECTION VIII — CONCLUSION
%% ======================================================================
\section{Conclusion}
\label{sec:conclusion}

This paper presented SAKE-IoT, a four-phase session amortization protocol
that bridges the efficiency gap between post-quantum cryptographic
strength and IoT operational feasibility. By extending the Ring-LWE
ring signature authentication and QC-LDPC code-based key encapsulation
of Kumari et~al.\ with a 1-RTT piggybacked handshake, an HKDF master
key extractor, and an AES-256-GCM amortized data layer, SAKE-IoT achieves
\emph{three independently validated performance improvements}: 99.7\%
end-to-end latency reduction, 45.1\% per-packet bandwidth saving, and
94.7\% session-level bandwidth saving at $N=20$ packets per epoch.

Three formal security proofs (IND-CCA2, replay resistance, Epoch-Bounded
Forward Secrecy) cover the entire session layer, and a complete
Contiki-NG/Cooja network emulation over IEEE~802.15.4 with 6LoWPAN and
RPL provides the first physical-layer validation of a post-quantum
session amortization protocol, demonstrating 100\% packet delivery ratio
and two verified epoch renewal cycles.

\noindent\textbf{Future Work:}
(1)~Multi-node simulations with 3--5 senders to evaluate anonymity set
effectiveness and hub scalability.
(2)~Time-based epoch expiry ($\Tmax$) in addition to packet-count expiry.
(3)~Port to real Z1 hardware (MSP430) using $n=32$ Ring-LWE parameters.
(4)~Energest radio and CPU energy profiling for direct mJ comparison.

%% ======================================================================
%% BIBLIOGRAPHY
%% ======================================================================
\section*{References}
\bibliographystyle{elsarticle-num}

\begin{thebibliography}{99}

\bibitem{kumari2022}
S. Kumari, M. Singh, R. Singh, H. Tewari,
``A post-quantum lattice-based lightweight authentication and
code-based hybrid encryption scheme for IoT devices,''
\textit{Computer Networks}, vol.\ 217, p.\ 109327, 2022.
\doi{10.1016/j.comnet.2022.109327}

\bibitem{nist_pqc2022}
NIST, ``Post-Quantum Cryptography Standardization,'' 2022.
\url{https://csrc.nist.gov/Projects/post-quantum-cryptography}

\bibitem{lyubashevsky2012}
V. Lyubashevsky, C. Peikert, O. Regev,
``On ideal lattices and learning with errors over rings,''
\textit{JACM}, vol.\ 60, no.\ 6, pp.\ 1--35, 2013.

\bibitem{rfc5869}
H. Krawczyk, P. Eronen,
``HMAC-based Extract-and-Expand Key Derivation Function (HKDF),''
RFC 5869, IETF, 2010.

\bibitem{nist38d}
NIST, ``Recommendation for Block Cipher Modes of Operation:
Galois/Counter Mode (GCM) and GMAC,'' NIST SP~800-38D, 2007.

\bibitem{rfc8446}
E. Rescorla,
``The Transport Layer Security (TLS) Protocol Version 1.3,''
RFC 8446, IETF, 2018.

\bibitem{pqctls2020}
D. Stebila, S. Fluhrer, S. Gueron,
``Hybrid key exchange in TLS 1.3,''
IETF Draft, 2020.

\bibitem{rfc7228}
C. Bormann, M. Ersue, A. Keranen,
``Terminology for Constrained-Node Networks,''
RFC 7228, IETF, 2014.

\bibitem{li2020}
X. Li et~al.,
``Lattice-based privacy-preserving and anonymous authentication
scheme,'' \textit{IEEE IoT Journal}, 2020.

\bibitem{wang2019}
H. Wang et~al.,
``Ring-LWE based two-factor authentication,''
\textit{IEEE Access}, 2019.

\bibitem{cheng2021}
Q. Cheng et~al.,
``Certificateless ring signature authentication for IoT,''
\textit{Sensors}, 2021.

\bibitem{lee2019}
J. Lee et~al.,
``RLizard: Post-quantum key encapsulation,''
\textit{IEEE Access}, 2019.

\bibitem{hu2019}
J. Hu et~al.,
``QC-LDPC key encapsulation on FPGA,''
\textit{IEEE Trans.\ VLSI}, 2019.

\bibitem{chikouche2020}
N. Chikouche, F. Khalfaoui,
``Code-based authentication for resource-constrained IoT,''
\textit{Computers \& Security}, 2020.

\bibitem{reza2021}
R. Reza et~al.,
``Lightweight IoT authentication with ECC session keys,''
\textit{IEEE IOTJ}, 2021.

\bibitem{farash2019}
M. Farash et~al.,
``Efficient session key derivation for IoT,''
\textit{Wireless Comms.}, 2019.

\end{thebibliography}

\end{document}
